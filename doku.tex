\documentclass[11pt,a4paper]{IEEEtran}
\usepackage[ngerman]{babel}
\usepackage[utf8]{inputenc}
\usepackage{enumitem} % referencable enumerations
\usepackage{eurosym}

\usepackage{siunitx}

% make inches available
\let\DeclareUSUnit\DeclareSIUnit
\let\US\SI
\DeclareUSUnit\inch{Zoll}

\usepackage{biblatex}
\addbibresource{references.bib}

\begin{document}

\title{DIY-Projekt Kaffeekasse}
\author{\IEEEauthorblockN{Martin Hofmann}\\
    \IEEEauthorblockA{Universität Erlangen-Nürnberg\\
    Email: martin.hofmann@fau.de}
}

\date{6. Februar 2016}

\maketitle
\begin{abstract}
    Blubbblabla
\end{abstract}

\section{Motivation}

Ziel des Projektes ist es, ein System zur Verwaltung von Nutzerkonten und zur
Abrechnung von 

Um die Abrechnung von Kaffeekäufen einfach zu gestalten, ergaben sich folgende
Anforderungen an das System:

\begin{enumerate}
    \item\label{req0} Die Authentifizierung von Nutzern muss einfach erfolgen.
    \item\label{req1} Die Bedienung des Gerätes muss intuitiv möglich sein.
    \item\label{req2} Das Gerät muss ohne Netzwerkverbindung funktionieren.
\end{enumerate}

Anforderung \ref{req0} soll durch die Verwendung von Firmenausweisen zur 
Identifizierung von Nutzern gelöst werden. Neben technischen Schwierigkeiten,
auf die im Kapitel \ref{sec:rfid} genauer eingegangen wird, ergaben sich hierbei
auch den Datenschutz betreffende Konsequenzen.

Beim Zugriff auf den Firmenausweis soll das Lesen von persönlichen
Informationen (Personalnummer, Schließberechtigungen und ähnliches)
unterbleiben. Da im Rahmen dieses Projektes jedoch nur die ID-Nummer des im
Ausweis eingebetteten Chips benötigt wird, welche nicht ohne weiteres der
Personalnummer zugeordnet werden kann, sind datenschutztechnische Belange
erfüllt.

Um eine einfache Bedienung des Gerätes nach Punkt \ref{req1} zu ermöglichen,
wurde entschieden, eine Touch-Oberfläche zu entwickeln. Auf Details hierzu
wird in den Kapiteln \ref{sec:hw} (der verwendete Touchscreen) und \ref{sec:sw}
 (Software für die grafische Oberfläche) eingegangen.

\section{Auslesen von RFID-Karten}
\label{sec:rfid}

\section{Hardware}
\label{sec:hw}

Dieses Kapitel gibt einen Überblick über die verwendeten Hardware-Komponenten.

\subsection{Raspberry Pi}

\subsection{Touchscreen}

Die geplante Anwendung hat geringe Anforderungen an einen Touch-Screen: Da nur
wenige Zeilen Text sowie Schaltflächen angezeigt werden müssen, sind auch 
Bildschirme mit einer kleinen Diagonale geeignet.

\subsubsection{Übersicht über verfügbare Optionen}

Für den Raspberry Pi sind verschiedenste Touch-Screen-Module verfügbar.
Besonders häufig vertreten sind hier Module mit einer Größe von circa
\SI{3}{\inch}. Mit Preisen von unter 20 \euro sind sie sehr günstig auf den
einschlägigen Handelsplattformen im Internet erhältlich.

Allen Angeboten gemeinsam ist, dass sie im kleingedruckten auf ausführliche
Installations-Anweisungen für Treiber verweisen. Diese enthalten entweder
zusätzliche Paketquellen mit angepassten Linux-Kernel-Images oder komplette
vorkonfigurierte Betriebssystem-Images. 

Der Ursprung dieser Software-Komponenten ist meist nicht bekannt, in den
Angeboten wird oft auf Downloads bei anonymen Cloud-Plattformen oder gar
Share-Hostern verwiesen. Da die Treiber oft nur in Binärform vorliegen, und 
die Update-Situation nicht geklärt ist, ist die Zukunftssicherheit solcher 
Lösungen ungewiss.

Da der Autor dieser Dokumentation mit der Plattform \emph{Cubietruck} bereits
schlechte Erfahrungen mit kaum gepflegten Binärtreibern gemacht hat, schieden
diese Touch-Screen-Module aus.

Seit Anfang 2015 ist ein offizielles Touch-Screen-Modul für den Raspberry Pi 
erhätlich, welches von der Raspberry Pi Foundation entwickelt wurde. Treiber
für diesen Bildschirm sind im Raspbian-Betriebssystem-Image bereits enthalten.

Im Unterschied zu den vorgenannten Modulen ist das offizielle Modul deutlich 
größer (\SI{7}{\inch} Diagonale) und auch teurer (circa 60 \euro). Aufgrund
der einfacheren Installation und der Verfügbarkeit von Updates wurde entschieden,
das offizielle Display zu verwenden.

\subsubsection{Bewertung}

Das Display lief wie versprochen mit einem aktuellen Raspbian-Image ohne
jegliche weitere Konfiguration. Auffällig ist die geringe Blickwinkelstabilität
 -- die Qualität ist unter der von günstigen Tablet-Computern. 

Als weiterer Negativ-Punkt ist die fehlende Unterstützung der
Helligkeitsregelung zu nennen. Während die Beleuchtung nach einem Patch von
Software ein- und ausgeschaltet werden kann, ist die Erweiterung um eine
Helligkeitsregelung ungewiss\autocite{TouchBacklight}.

\subsection{DS3231 Uhr-Modul}

Der Raspberry Pi enthält keinen RTC-Chip\footnote{\emph{real-time clock},
speichert physikalische Zeit}. Bei jedem Systemstart wird die Zeit per
NTP\footnote{\emph{network time protocol}} aus dem Internet bezogen.

Da im geplanten Einsatz-Szenario keine Internet-Verbindung besteht, musste ein
externes RTC-Modul für den Raspberry Pi nachgerüstet werden. Es enthält eine
auswechselbare Puffer-Batterie, damit die Uhrzeit auch bei Entfernen der 
Stromquelle des Raspberry Pi aktuell bleibt. 

%TODO bild

Angebunden wird das Modul über den I2C-Bus.

\section{Software}
\label{sec:sw}

\subsection{Der Adafruit-PN532-Python-Treiber} %\end{document}  % This is where a 'short' article might terminate

\subsection{Grafische Oberfläche}
\label{sec:gui}

\subsection{Datenhaltung}

\subsection{Modellierung als Zustandsautomat}

\section{Aktueller Stand}

\section{Ausblick und Erweiterungsmöglichkeiten}

Mehrfach wurde vorgeschlagen, zur einfacheren Authentifizierung von Nutzern 
einen Fingerabdruckleser zu verwenden. 

Ein weiteres Wunsch-Feature ist die Abspeicherung von Transaktionen mit
Zeitstempeln. Dies würde eine Analyse des persönlichen Kaffee-Konsums
ermöglichen. Die Implementierung dieses Features sollte als Opt-In nur für
interessierte Benutzer erfolgen. 

\printbibliography
\end{document}
